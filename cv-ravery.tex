% Danny Avery - Curriculum Vitae
% Copyright 2019 Danny Avery
% Email: dcavery@ucsc.edu
% Web: 

\documentclass[12pt,letterpaper]{report}

\usepackage[T1]{fontenc} % output T1 font encoding (8-bit) so accented characters are a single glyph
\usepackage[utf8]{inputenc} % allow input of utf-8 encoded characters
\usepackage[strict,autostyle]{csquotes} % smart and nestable quote marks
\usepackage[USenglish]{babel} % automatically regionalize hyphens, quote marks, etc
\usepackage{microtype} % improves text appearance with kerning, etc
\usepackage{datetime} % enable formatting of date output
\usepackage{tabto} % make nice tabbing
\usepackage{hyperref} % enable hyperlinks and pdf metadata
\usepackage{geometry} % manually set page margins
\usepackage{enumitem} % enumerate with [resume] option
\usepackage{titlesec} % allow custom section fonts

% what is your name?
\newcommand{\myname}{Danny Avery}

% define a default font face and set it as the body font
\usepackage{crimson} % document's serif font
\usepackage{helvet}  % document's sans serif font

% define how far to tab for list items with left-aligned date - different font faces need different widths
\newcommand{\listtabwidth}{1.75cm}

% set name font to title the document
\newcommand{\namefont}[1]{{\normalfont\bfseries\Huge{#1}}}

% set section heading fonts and before/after spacing
\SetTracking{encoding=*}{20}
\titleformat{\section}{\sffamily\small\bfseries\lsstyle\uppercase}{}{}{}{}
\titlespacing{\section}{0pt}{24pt plus 4pt minus 2pt}{12pt plus 2pt minus 2pt}

% set subsection heading fonts and before/after spacing
\titleformat{\subsection}{\sffamily\footnotesize\bfseries}{}{}{}{}
\titlespacing{\subsection}{0pt}{12pt plus 4pt minus 2pt}{8pt plus 2pt minus 2pt}

% set page margins
\geometry{body={6.5in, 9.0in},
	left=1.0in,
	top=1.0in}

% prevent paragraph indentation
\setlength\parindent{0em}

% define space between list items
\newcommand{\listitemspace}{0.15em}

% make unordered lists without bullets and use compact spacing
\renewenvironment{itemize}
{\begin{list}{}{\setlength{\leftmargin}{0em}
			\setlength{\parskip}{0em}
			\setlength{\itemsep}{\listitemspace}
			\setlength{\parsep}{\listitemspace}}}
	{\end{list}}

% make tabbed lists so content is left-aligned next to years
\TabPositions{\listtabwidth}
\newlist{tablist}{description}{3}
\setlist[tablist]{leftmargin=\listtabwidth,
	labelindent=0em,
	topsep=0em,
	partopsep=0em,
	itemsep=\listitemspace,
	parsep=\listitemspace,
	font=\normalfont}

% print the month and year only when using \today
\newdateformat{monthyeardate}{\monthname[\THEMONTH] \THEYEAR}

% define hyperlink appearance and metadata for pdf properties
\hypersetup{
	colorlinks = true,
	urlcolor = black,
	pdfauthor = {\myname},
	pdfkeywords = {FPGA, IoT, Hydroponics, Rasperry Pi, Microcontroller},
	pdftitle = {\myname: Curriculum Vitae},
	pdfsubject = {Curriculum Vitae},
	pdfpagemode = UseNone
}

\begin{document}
	\raggedright
	
	% display name as the document title
	\namefont{\myname}
	
	% contact info
	\vspace{1em}
	\begin{minipage}[t]{0.495\textwidth}
		Department of Computer Science and Engineering \\
		University of California, Santa Cruz \\
		Santa Cruz, California
	\end{minipage}
	\begin{minipage}[t]{0.495\textwidth}
		Email: dcavery@ucsc.edu \\
		%Web: \href{http://caylor.eri.ucsb.edu/people/avery/}{http://caylor.eri.ucsb.edu/people/avery/} \\
		Phone: +1 916 897 7713
	\end{minipage}
	\vspace{0.5em}
	
	
	
	\section*{Education}
	
	\begin{tablist}
		
		\item[B.S.]  \tab Computer Science and Engineering: Computer Systems Concentration, University of California, Santa Cruz, 2020
		
	\end{tablist}	


	
	\section*{Research Interests}
	
	\begin{itemize}
		
		\item Ecohydrological measurement and modeling, vegetation remote sensing, isotope hydrology
		
		\item Classification and clustering of multispectral imagery for object detection and segmentation
		
		\item Spatial data science, machine learning, big data, visualization, spatial analysis
		
		\item Coupled human-natural systems and water resources management 
		
	\end{itemize}
	
	
	
    \section*{Publications}
	
    \subsection*{Manuscripts in Preparation}

    \begin{tablist}

        \item[\the\year] \tab Avery, R., K. Caylor, M. McCabe, M. Mayes, L. Estes \enquote{Field-Scale Maps of Evapotranspiration across Center Pivot Agriculture in Drylands} Target: \textit{Remote Sensing of Environment}, Fall 2019.

        \item[\the\year] \tab Avery, R., L. Estes, K. Caylor, S., R. Eastman, Ye, L. Song \enquote{A Convolutional Neural Network Approach for Segmenting Smallholder Agriculture and Comparison to Modern Machine Learning Methods} Target: \textit{Remote Sensing of Environment}, Summer 2019.        

        \item[\the\year] \tab Tuholske, C., K. Caylor, T. Evans, R. Avery \enquote{Triangulating Urban Agglomerations Hotspots Across Africa} Target: \textit{Environmental Research Letters}, Spring 2019.

        \item[\the\year] \tab Elmes A., L. Estes, M. Friedl, V. Gammino, J. McCarty, M. Jain, L. Fishgold, K. Caylor, R. Eastman, G. Pontius, J. Bayas, H. Alemohammad, J. Rogan, D. Kohli, R. Avery, D. Lunga, I. Bouvier   \enquote{Quantifying Error in Training Data and its Implications for Land Cover Mapping} Target: \textit{Remote Sensing of Environment}, Summer 2019.
    
    \end{tablist}


    \subsection*{Reports}

	\begin{tablist}
		
		\item[2017] \tab \enquote{Detecting Changes in Nighttime Sky Brightness over Grand Teton National Park with the Suomi NPP VIIRS Sensor} Avery, R., V. Warda, S. Chu, S. Chao. 2017. NASA DEVELOP Technical Report. 

	   	\item[2017] \tab \enquote{Enhancements to Visualization of CALIPSO (VOCAL) through Case Studies of Saharan Dust} Pampalone, C. R. Avery, W. Turner. 2017. NASA DEVELOP Technical Report.

       	\item[2017] \tab \enquote{A Threshold-Based Decision Tree Approach to Mapping Landscape Disturbance in Glacier National Park} Avery, R., Mays, C., Alvarado A. 2017. NASA DEVELOP Technical Report.

        \item[2016] \tab \enquote{Mapping Invasive Species to Efficiently Monitor Southwestern National Park Areas} Avery, R., K. Landesman, T. Whaley. 2016. NASA DEVELOP Technical Report.
	
    \end{tablist}

	
	\subsection*{Conference Presentations}
	
	\begin{tablist}
		
		\item[2018] \tab Avery, R., \enquote{A Convolution Neural Network Approach for Segmenting Center Pivot Agriculture} American Geophysical Union Fall Meeting. Washington D.C. Dec 10--14.
						
	\end{tablist}
	
	
	
	\section*{Grants and Awards}
	
	\subsection*{Grants and Fellowships}
	
	\begin{tablist}
		
        \item[2019] \tab Honorable Mention, National Science Foundation Graduate Research Fellowship Program.
        \item[2019] \tab Scipy 2019 Scholarship Award, Full Conference Scholarship (\$1903).
        \item[2019] \tab Travel Scholarship to attend Isocamp 2019 (\$1000).
        \item[2019] \tab Travel Scholarship to AI for Earth Summit 2019 (\$1500 approx.).
		\item[2018] \tab National Geographic and Microsoft AI for Earth Innovation research grant (\$100,000). Role: Primary Author and Project Member.
		
	\end{tablist}
	


    \section*{Research Experience}

    \begin{tablist}

        \item[January 2019 -- present] \tab National Geographic AI for Earth Fellowship, Primary Researcher and Project Team Member. University of California, Santa Barbara.

        \item[January 2018 -- present] \tab Clark Labs, Graduate Research Assistant. Worcester, Massechusetts; 

        \item[September 2016 -- August 2017] \tab NASA DEVELOP National Program, Geoinformatics and Project Coordination Fellow. NASA Langley Research Center, Virginia.

        \item[June 2016 -- August 2016] \tab NASA DEVELOP National Program, Team Lead and Researcher. NASA Langley Research Center, Virginia.

        \item[May 2015 -- December 2015] \tab Berkeley Energy and Climate Institute, Undergraduate Research Fellow. University of California, Berkeley.

        \item[September 2014 -- April 2015] \tab Kelly Research and Outreach Lab, Undergraduate Researcher. University of California, Berkeley.

    \end{tablist}

	
	
	\section*{Teaching Experience}
	
	\subsection*{University of California, Santa Barbara}
	
	\begin{itemize}
		
        \item Oceans and Atmosphere, Teaching Assistant. (Winter '19)

		\item Oceans and Atmosphere, Teaching Assistant. (Fall '18)
		
	\end{itemize}
	
	
	
	\section*{Service}
	
	\subsection*{Service Workshops}
	
	\begin{itemize}
		
        \item The Unix Shell, Git/Github, Python, Center for Scientific Computing, May 11--12, 2019 (upcoming)

        \item Jupyter Notebooks and Python for Ecologists, EcoDataScience at UCSB, November 13, 2018

        \item The Unix Shell, Git/Github, R for Reproducible Scientific Analysis, Old Dominion University, October 25--26, 2018

        \item The Unix Shell, Git/Github, Python, CSU Monterey Bay, January 19--20, 2018

        \item The Unix Shell, Git/Github, Batch Processing with GDAL, NASA JPL, September 18--19, 2017

        \item The Unix Shell, Git/Github, Python, NASA DEVELOP at Wise County, June 12--13, 2017

        \item The Unix Shell, Git/Github, Python, NASA Langley Research Center, June 8--9, 2017
		
        \item Programming with Python, NASA Langley Research Center, January 26--27, 2017 
		
	\end{itemize}
	
	\subsection*{Service to Department}
	
	\begin{itemize}
		
        \item Geography Ph.D. program faculty representative, University of Santa Barbara, 2018--19

		\item Computing Resources Committee, University of Santa Barbara, 2017--2018
		
	\end{itemize}
	
	
	
	\section*{Professional Affiliations}
	
	\begin{itemize}
		
		\item American Geophysical Union
		
		\item The Carpentries (Software and Data Carpentry)

                \item National Geographic Explorers
		
	\end{itemize}
	
	
	
	\section*{Credentials}
	
	\begin{itemize}
		
		\item Certified Instructor for Software and Data Carpentry, including geospatial data science lessons
		
	\end{itemize}
		
	
	
	\section*{Selected Media Coverage}
	
	\begin{tablist}
		
		\item 2019 \tab \textit{The UCSB Current}. \enquote{Eyes in the Sky: National Geographic awards geographer Kelly Caylor an 'AI for Earth Innovation' grant} January 29, 2019.
		
		\item 2018 \tab \textit{southbigdatahub.org}. \enquote{Old Dominion University: A Melting Pot of Learners and Perspectives Creates an Impactful Workshop} October 27, 2018.
				
	\end{tablist}
	
	
	
	\section*{Skills and Methods}
	
	\subsection*{Statistical and Computational Methods}
	
	\begin{itemize}
		
		\item Computational statistics and machine learning, radiometric and atmospheric calibration of multispectral and hyperspectral imagery, evapotranspiration estimation (leaf to canopy scales), water balance modeling, isotope mixing models

                \item Data mining, data wrangling, Python (including numpy, scipy, pandas, matplotlib, statsmodels, scikit-learn, and scikit-image), deep learning (including keras, imgaug, and mrcnn), Apache Spark, Amazon Web Services, Microsoft Azure, JavaScript, HTML, MySQL.
		
	\end{itemize}
	
	\subsection*{Geospatial Methods and Tools}
	
	\begin{itemize}
		
		\item ENVI, MODTRAN, geopandas, rasterio, rasterstats, rasterfames, geopyspark, rastervision, Planet Labs API, spatial analysis, QGIS, GRASS GIS, ArcGIS, Leaflet
		
	\end{itemize}
	
	\subsection*{Field Methods}
	
	\begin{itemize}
		
		\item I have experience with drone remote sensing for digital terrain modeling based on structure from motion, LICOR infrared gas analyzers, thermal radiometers, plot level plant morphology measurements, biomass weighing, and geolocating transect and point data. I also have experience with taking cores of riparian trees, measuring leaf level transpiration and carbon assimilation with LI-COR instruments, collecting water samples and tree cores for isotopic analysis, auguring for soil samples and taking soil moisture profile measurements.
		
	\end{itemize}
	
	
	
	% display today's date as Month Year after a vertical space below the end of the text
	\begin{center}
		\vspace{6em}
		%\vfill
		Updated \monthyeardate\today
	\end{center}
	
	
	
\end{document}
